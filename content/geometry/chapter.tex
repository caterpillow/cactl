\chapter{Geometry}

\section{Notes}
	For checking imprecise equalities use \texttt{abs(x) < eps}. \\
	Clamp values to $[-1, 1]$ for \texttt{acos}, etc.

\section{Precision}
	\textbf{double} can precisely store integers up to $9 \times 10^{15}$, and \textbf{long double} can store up to $1.8 \times 10^{19}$. \\	
	Operations $(+, \times, -, /, \sqrt{x})$ will be rounded to the nearest representable value. Hence, the \textbf{relative} error on the result of one of these operations is bounded by $\epsilon = 1.2 \times 10^{-16}$ for \textbf{double} and $\epsilon = 5.5 \times 10^{-20}$ for \textbf{long double}. \\
	If computations are precise up to a relative error of $\epsilon$, and the magnitude of our values never exceeds $M$, then the absolute error of an operation is at most $M \epsilon$. \\
	It is typically \textbf{more useful to consider absolute error} instead of relative error (1). \\

	A series of $n$ $+$ and $-$ operations result in an absolute error of at most $nM\epsilon$. \\
	With $\times$, the absolute error of a $d$-dimensional value computed in $n$ operations is $M^{d}((1+\epsilon)^{n}-1) \approx nM^{d}\epsilon$ (excluding multiplication by adimensional values). \\
	Imprecisions on $x$ in $\frac{1}{x}$ can cause an arbitrary error (2). \\
	Imprecisions on $x$ in $\sqrt{x}$ are typically bad, particularly near $0$. (3) \\

	Note that $\frac{1}{x}$ and $\sqrt{x}$ perform poorly on imprecise inputs: when working with exact inputs, the \texttt{IEEE 754} standard guarantees a relative error of $\epsilon$/absolute error of $M\epsilon$. \\ 
	In particular, circle/line/circle-line intersections on exact coordinates have relative error proportional to $\epsilon$/absolute error proportional to $M\epsilon$.

	(1) subtracting two large imprecise values can blow up relative errors \\
	(2) division by a value near $0$ can result in an arbitrarily large error in both positive and negative \\
	(3) assuming arguments to $\sqrt{x}$ are non-negative, imprecisions on $x$ will have the most impact when $x$ is near $0$: for a given imprecision $\delta$, the biggest imprecision on $\sqrt{x}$ it might cause is $\sqrt{\delta}$. This is quite bad: an argument with imprecision $nM^2\epsilon$ will become an imprecision of $\sqrt{n}M\sqrt{\epsilon}$. This can appear with circle intersections and computing roots to quadratics. \\

	\kactlimport{StableSum.h}

\section{Common}
	\subsection{Equation of line through two points}
		Given points $p$ and $q$, equation of the line $ax + by + c = 0$ is given by:
		\small
		\[ a = p_y - q_y \]
		\[ b = q_x - p_x \]
		\[ c = p \times q \]
		\normalsize

\section{Geometric primitives}
	\kactlimport{Point.h}
	\kactlimport{lineDistance.h}
	\kactlimport{SegmentDistance.h}
	\kactlimport{SegmentIntersection.h}
	\kactlimport{lineIntersection.h}
	\kactlimport{sideOf.h}
	\kactlimport{OnSegment.h}
	\kactlimport{linearTransformation.h}
	\kactlimport{LineProjectionReflection.h}
	\kactlimport{Angle.h}

\section{Circles}
	\kactlimport{CircleIntersection.h}
	\kactlimport{CircleTangents.h}
	\kactlimport{CircleLine.h}
	\kactlimport{CirclePolygonIntersection.h}
	\kactlimport{circumcircle.h}
	\kactlimport{MinimumEnclosingCircle.h}

\section{Polygons}
	\kactlimport{InsidePolygon.h}
	\kactlimport{PolygonArea.h}
	\kactlimport{PolygonCenter.h}
	\kactlimport{PolygonCut.h}
	\kactlimport{PolygonUnion.h}
	\kactlimport{ConvexHull.h}
	\kactlimport{HullDiameter.h}
	\kactlimport{PointInsideHull.h}
	\kactlimport{LineHullIntersection.h}
	\kactlimport{MinkowskiSum.h}

\section{Misc. Point Set Problems}
	\kactlimport{ClosestPair.h}
	\kactlimport{ManhattanMST.h}
	\kactlimport{kdTree.h}
	% \kactlimport{DelaunayTriangulation.h}
	\kactlimport{FastDelaunay.h}

\section{3D}
	\kactlimport{PolyhedronVolume.h} % bruh
	\kactlimport{Point3D.h} % bruh
	\kactlimport{3dHull.h} % bruh
	\kactlimport{sphericalDistance.h}
